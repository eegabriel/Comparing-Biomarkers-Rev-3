% !TEX TS-program = pdflatex
% !TEX encoding = UTF-8 Unicode

% This is a simple template for a LaTeX document using the "article" class.
% See "book", "report", "letter" for other types of document.

\documentclass[11pt]{article} % use larger type; default would be 10pt

\usepackage[utf8]{inputenc} % set input encoding (not needed with XeLaTeX)

%%% Examples of Article customizations
% These packages are optional, depending whether you want the features they provide.
% See the LaTeX Companion or other references for full information.

%%% PAGE DIMENSIONS
\usepackage{geometry} % to change the page dimensions
\geometry{a4paper} % or letterpaper (US) or a5paper or....
% \geometry{margin=2in} % for example, change the margins to 2 inches all round
% \geometry{landscape} % set up the page for landscape
%   read geometry.pdf for detailed page layout information

\usepackage{graphicx} % support the \includegraphics command and options

% \usepackage[parfill]{parskip} % Activate to begin paragraphs with an empty line rather than an indent

%%% PACKAGES
\usepackage{booktabs} % for much better looking tables
\usepackage{array} % for better arrays (eg matrices) in maths
\usepackage{paralist} % very flexible & customisable lists (eg. enumerate/itemize, etc.)
\usepackage{verbatim} % adds environment for commenting out blocks of text & for better verbatim
\usepackage{subfig} % make it possible to include more than one captioned figure/table in a single float
% These packages are all incorporated in the memoir class to one degree or another...

%%% HEADERS & FOOTERS
\usepackage{fancyhdr} % This should be set AFTER setting up the page geometry
\pagestyle{fancy} % options: empty , plain , fancy
\renewcommand{\headrulewidth}{0pt} % customise the layout...
\lhead{}\chead{}\rhead{}
\lfoot{}\cfoot{\thepage}\rfoot{}

%%% SECTION TITLE APPEARANCE
\usepackage{sectsty}
\allsectionsfont{\sffamily\mdseries\upshape} % (See the fntguide.pdf for font help)
% (This matches ConTeXt defaults)

%%% ToC (table of contents) APPEARANCE
\usepackage[nottoc,notlof,notlot]{tocbibind} % Put the bibliography in the ToC
\usepackage[titles,subfigure]{tocloft} % Alter the style of the Table of Contents
\renewcommand{\cftsecfont}{\rmfamily\mdseries\upshape}
\renewcommand{\cftsecpagefont}{\rmfamily\mdseries\upshape} % No bold!

%%% END Article customizations

%%% The "real" document content comes below...

\title{Response to Referee’s report of SIM submission 13-0382.R1 entitled "Comparing and
combing biomarkers as principle surrogates for time-to-event clinical
endpoints"}
\author{Erin E. Gabriel, Michael C. Sachs, Peter B. Gilbert}
%\date{} % Activate to display a given date or no date (if empty),
         % otherwise the current date is printed 

\begin{document}
\maketitle
We are happy to have the opportunity to revise our manuscript. We have put forth considerable effort to address reviewer 2's comments and we feel that the paper has improved greatly. 

We have made some critical revisions to the methods section, and have modified the real-data example to better illustrate our proposed methods.  We have developed a new summary statistic that does not require the selection of a specific time. During the development of this new summary statistic, we noticed a convergence issue for the multivariate PS estimands, that can lead to poor efficiency in practice. This issue was previously masked by our truncation of the STG for a given time point. We now consider using a multivariate BIP, one independent BIP for each candidate PS, that appears to solve the convergence issues. 

We have spent several months attempting to locate an adequate new example data set. Although we were unable to do so, we tried several, including the Mr. FIT data, and several unpublished small immune marker trial of NIAID. In each case there was inadequate baseline information to support the estimation of the counterfactual estimands. However, in running each of these examples, we determined that the use of truncated normal in the simulations did not fit what we were suggesting in practice for the example. We have modified these simulations to use censored normal generation and censored imputation. Below we outline the specific changes. 


\section{Comments from Reviewer 2}
\begin{quotation}
I have to admit that I am somewhat disappointed with this revision.

Only minor changes in the essential section 2 have been made.

The paper is still not sufficiently self-contained and one has to read at least [1] and [10] to have a chance at understanding what’s going on in the paper. Both [1] and [10] are very nice and interesting papers by the way.

The extension from [10] in terms of modelling is straightforward and minor.

The tools developed to assess and select one or more surrogates are problematic at least to me as they include the subjective choice of one or more time points at which performance is evaluated. Also, the actual selection of an optimal subset of surrogates is not well described.

It would be ideal with an example where these extensions really mattered.
\end{quotation}

\section{Response to Reviewer 2}
\begin{enumerate}
\item We reorganized the methods into sections 2 and 3. Section 2 now contains the notation and background information on the concept of principal surrogacy. This section provides a more thorough introduction to the methods so that uninitiated readers may better understand without referring to [1] or [10]. 
\item The remaining subsections of the methods have been reorganized so that the concepts flow more naturally from defining estimands of interest, to assumptions, to modeling, to estimation, to summary statistics, and finally to PS comparison. 
\item We have added an additional subsection where we introduce the integrated standardized total gain to address reviewer 2's concerns about the subjective choice of time points.  The integrated STG does not require the analyst to specify a time point, but rather averages the time dependent STG with respect to the distribution of event times. We suggest estimating the marginal distribution of event times with the Kaplan-Meier estimator. We evaluate the integrated STG in the simulations and demonstrate its operating characteristics. The integrated STG tends to have slightly higher power for comparison of candidate PS as compared to the STG at a single time point. 
\item In regard to the reviewer's comment about selecting the optimal subset of surrogates, we have revised section 3.7 to more clearly describe the hypothesis tests that are used to rule out useless PS from consideration and for comparing and ranking candidates with some value. 
\item We have modified the example to better illustrate our extensions. We created a third candidate PS from a linear combination of the two real candidate PS such that the PS has some value as a surrogate, but with a significantly smaller STG. In this case the test for the difference in STG yields evidence of a difference between the two surrogates where standard hypothesis testing could not. This makes the example more like a simulation, in that we modified the data to generate the desired results, but we feel that it does provide a better illustration of where our proposed methods actually matter. 
\item We have added to the multivariate simulations the investigation of the use of a multivariate BIP. 
\item We have added updated the simulations to reflect our suggested censoring of a candidate surrogate, as done in the example, rather than truncation. 
\end{enumerate}

We thank you again for the thoughtful review; the paper has improved greatly thanks to your efforts. 

\vspace{2em}

\hfill{Sincerely,}
\vspace{2em}

\hfill{Erin E. Gabriel}


\end{document}
